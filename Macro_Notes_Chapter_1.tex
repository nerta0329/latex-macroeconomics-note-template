\documentclass[11pt, a4paper]{article}

%----------------------------------------------------------------------------------------
%	1. 基础配置与宏包
%----------------------------------------------------------------------------------------
\usepackage[fontset=ubuntu]{ctex} % 中文支持
\usepackage[margin=1in]{geometry} % 页边距
\usepackage{amsmath, amsthm, amssymb} % 数学公式
\usepackage{mathrsfs} % 花体字
\usepackage{booktabs} %以此制作专业表格
\usepackage{enumitem} % 列表定制
\usepackage{fancyhdr} % 页眉页脚
\usepackage{xcolor}   % 颜色支持
\usepackage[explicit]{titlesec} % 标题格式
\usepackage{setspace} % 行间距
\usepackage{parskip}  % 段落间距

% 设置正文行间距 (1.4倍,更利于阅读)
\linespread{1.4}

%----------------------------------------------------------------------------------------
%	2. 配色方案 (Morandi / Modern Textbook)
%----------------------------------------------------------------------------------------
\definecolor{primaryColor}{HTML}{003f5c}   % 主色:深蓝 (用于标题、强调)
\definecolor{accentColor}{HTML}{bc5090}    % 强调色:紫红 (用于关键变量)
\definecolor{boxBack}{HTML}{f0f5f9}        % 盒子背景:极淡蓝
\definecolor{alertBack}{HTML}{fff2f2}      % 警告背景:极淡红
\definecolor{alertFrame}{HTML}{ff6361}     % 警告边框:珊瑚红

%----------------------------------------------------------------------------------------
%	3. 标题样式定制 (核心修复部分)
%----------------------------------------------------------------------------------------
\usepackage[most]{tcolorbox}

% 3.1 带编号的一级标题 (例如: 1. 宏观概览)
% 修复逻辑:强制编号和标题在同一行显示
\titleformat{name=\section}[block]
  {\normalfont\Large\bfseries\color{primaryColor}}{}{0em}{%
  \begin{tcolorbox}[
      enhanced,
      colback=primaryColor!10,      % 浅色背景
      colframe=primaryColor,        % 深色边框颜色
      leftrule=4mm,                 % 左侧粗竖条
      rightrule=0mm, toprule=0mm, bottomrule=0mm, % 隐藏其他边框
      arc=0mm,                      % 直角
      boxsep=5pt,
      left=10pt,
      width=\dimexpr\textwidth\relax,
      nobeforeafter,                % 不强制换行
      valign=center                 % 垂直居中
    ]
    \thesection\quad #1             % 核心:数字+空格+标题
  \end{tcolorbox}%
}

% 3.2 无编号的一级标题 (例如: 目录)
% 修复逻辑:目录不显示编号,也不使用色块框,保持简洁
\titleformat{name=\section,numberless}[block]
  {\normalfont\Large\bfseries\color{primaryColor}}{}{0em}{#}

% 3.3 二级标题样式 (带下划线)
\titleformat{\subsection}
  {\normalfont\large\bfseries\color{primaryColor}}
  {\thesubsection}{0.5em}
  {#1}
  [\vspace{-0.3em}\color{primaryColor!30}\hrule height 1pt]

%----------------------------------------------------------------------------------------
%	4. 内容盒子设计
%----------------------------------------------------------------------------------------

% 核心理论盒子 (蓝色系)
\newtcolorbox{theorybox}[2][]{%
  enhanced,
  colback=boxBack,
  colframe=primaryColor,
  coltitle=white,
  fonttitle=\bfseries\sffamily,
  title={#2},
  attach boxed title to top left={yshift*=-\tcboxedtitleheight/2, xshift=5mm},
  boxed title style={colback=primaryColor, rounded corners=2pt},
  drop fuzzy shadow,
  #1
}

% 批判/注意盒子 (红色系)
\newtcolorbox{alertbox}[2][]{%
  enhanced,
  colback=alertBack,
  colframe=alertFrame,
  coltitle=white,
  fonttitle=\bfseries\sffamily,
  title={#2},
  attach boxed title to top left={yshift*=-\tcboxedtitleheight/2, xshift=5mm},
  boxed title style={colback=alertFrame, rounded corners=2pt},
  drop fuzzy shadow,
  #1
}

% 补充定义盒子 (绿色系 - 对应模型分类)
\definecolor{defGreen}{HTML}{2d6a4f}
\definecolor{defBack}{HTML}{d8f3dc}
\newtcolorbox{defbox}[1]{%
  enhanced,
  colback=defBack,
  colframe=defGreen,
  coltitle=white,
  fonttitle=\bfseries\sffamily,
  title={#1},
  attach boxed title to top left={yshift*=-\tcboxedtitleheight/2, xshift=5mm},
  boxed title style={colback=defGreen, rounded corners=2pt},
  #1
}

%----------------------------------------------------------------------------------------
%	5. 页眉页脚
%----------------------------------------------------------------------------------------
\pagestyle{fancy}
\fancyhf{}
\fancyhead[L]{\sffamily\color{gray} Advanced Macroeconomics Notes}
\fancyhead[R]{\sffamily\color{gray} Chapter 1}
\fancyfoot[C]{\thepage}
\renewcommand{\headrulewidth}{0pt} 

%----------------------------------------------------------------------------------------
%	文档开始
%----------------------------------------------------------------------------------------
\begin{document}

% 封面区域
\begin{center}
    \vspace*{2cm}
    {\Huge \bfseries \color{primaryColor} Advanced Macroeconomics Notes}\\[0.8cm]
    {\Large \itshape \color{gray} Chapter 1: Macroeconomics Introduction}\\[2cm]
    
    \begin{tcolorbox}[colback=gray!10, colframe=gray!10, width=0.7\textwidth]
        \centering \sffamily Nerta | Fall 2025 \\[0.5em] % 编写者
        \textcolor{gray!80!black}{\footnotesize 编写于 2025 年秋季学期,仅供学习与交流参考,请勿用于商业用途。} % 声明
    \end{tcolorbox}
\end{center}

% 目录页
\tableofcontents
\newpage

%----------------------------------------------------------------------------------------
%	正文内容
%----------------------------------------------------------------------------------------

\section{宏观经济学概览 (Macroeconomics: A Tour)}

宏观经济学关注总量水平(Aggregate level)上的广泛经济问题。

\subsection{主要变量与现象}
宏观分析主要围绕以下核心变量与现象展开:
\begin{itemize}[itemsep=2pt]
    \item \textbf{主要变量 (Main Variables):} Output (产出), Unemployment (失业), Inflation rate (通胀率)。
    \item \textbf{主要现象 (Main Phenomena):}
    \begin{itemize}
        \item \textbf{Business Cycle (商业周期/经济波动):} 经济活动的短期震荡。
        \item \textbf{Long-run Growth (长期增长):} 产出的长期上升趋势。
    \end{itemize}
\end{itemize}

\subsection{核心问题与研究视角 (Horizons)}
依据时间跨度不同,宏观经济学的核心问题与研究重点有所区别:

\begin{itemize}[itemsep=2pt]
    \item \textbf{Short-run (短期):} 决定经济扩张与衰退的因素是什么?
    \item \textbf{Middle-run (中期):} 通胀、产出与失业之间的结构性关系是什么?
    \item \textbf{Long-run (长期):} 决定经济增长的根本因素是什么?
\end{itemize}

\subsubsection*{长短期研究视角对比}
\begin{center}
\renewcommand{\arraystretch}{1.4}
\begin{tabular}{p{0.15\textwidth} p{0.25\textwidth} p{0.15\textwidth} p{0.35\textwidth}}
\toprule
\textbf{Horizon} & \textbf{Topic} & \textbf{Dynamics} & \textbf{Stochastic Model} \\
\midrule
Short-Run & Business Cycle (BC) & Discrete & Stochastic GE (DSGE) \\
Mid/Long-Run & Growth & Continuous & Deterministic PE \\
\bottomrule
\end{tabular}
\end{center}

\begin{defbox}{模型分类说明}
\begin{itemize}
    \item \textbf{DSGE (Discrete Stochastic General Equilibrium):} 离散随机一般均衡模型,现代宏观主流范式。
    \item \textbf{Reduced Form (简化式):} 实际上通常是一个 PE (Partial Equilibrium) 模型,只关心单一市场的出清问题。
    \item \textbf{Structural Model (结构化模型):} 考虑各个市场之间的相互作用,建立 GE 模型。
\end{itemize}
\end{defbox}

\subsection{主要政策 (Main Policy)}
宏观经济政策的核心问题在于:一个政策是否有效?其效应有多大?
\begin{itemize}
    \item \textbf{Monetary Policy (货币政策)}
    \item \textbf{Fiscal Policy (财政政策)}
\end{itemize}

\section{宏观思想史 (History of Macro Thoughts)}

\subsection{大萧条前 (Pre-Great Depression)}
在 1930 年代之前,宏观经济学尚未成为独立学科,古典主义思想 (Classical thoughts) 占据主导地位。

\textbf{代表人物与观点:}
\begin{itemize}
    \item \textbf{Adam Smith:} 提出 "Invisible hand" (看不见的手)。
    \item \textbf{Walras \& Pigou:} 认为价格调整会自动使需求等于供给 (Market Clearing);经济倾向于充分就业 (Full Employment)。
    \item \textbf{Say’s Law (萨伊定律):} 供给创造需求。
\end{itemize}
\textbf{核心思想:} 市场是万能的,不需要政府干预。

\subsection{凯恩斯革命 (Keynesian Revolution)}
背景: 1930 年代大萧条 (Great Depression)。美英出现高失业率 (US 1933: 25\%),实际 GDP 剧烈下滑 (US 1933 vs 1929: -31\%)。

Keynes 代表作: \textit{The General Theory of Employment, Interest and Money}.

\begin{theorybox}{凯恩斯主义核心理论}
\begin{itemize}
    \item \textbf{工资刚性 (Rigid Wages):} 工资向下调整缓慢,导致劳动力供给不等于劳动力需求 ($L^S \neq L^D$),市场无法自动出清。
    \item \textbf{有效需求:} 总需求 (Aggregate demand) 对决定产出至关重要。
    \item \textbf{政策主张:} 由于市场无法自动修复,需要利用\textbf{财政政策}增加需求。
\end{itemize}
\end{theorybox}

古典与凯恩斯的区别:
\begin{itemize}
    \item \textbf{Classical:} 供给最重要。
    \item \textbf{Keynes:} 需求最重要,因为工资刚性导致市场无法自动出清,需政府干预。
\end{itemize}

模型发展:
\begin{itemize}
    \item \textbf{IS-LM Model (Hicks, 1937):} 将凯恩斯的理论形式化。
    \item \textbf{菲利普斯曲线 (Phillips Curve, PC):} 发现通胀与失业之间存在权衡 (Trade-off)。公式为 $\pi_t = -\alpha u_t$。
\end{itemize}

局限: 解决了通胀解释问题,但无法解释 70 年代的滞胀 (Stagnation)。由此发展出 附加预期的菲利普斯曲线:$\pi_t - \pi_t^e = -\alpha(u_t - u_t^n)$。

\subsection{新古典学派 (Neoclassical)}
核心特征: 强调微观基础 (Micro foundation) 和 理性预期 (Rational expectations)。

\subsubsection{1. 货币主义 (Monetarism)}
Friedman \& Schwartz (1963) 认为美联储错误的货币政策是大萧条的主要原因。
\begin{itemize}
    \item \textbf{政策主张:} 常数货币供给增长 (Constant money supply growth) 是稳定经济的最佳方式。
    \item \textbf{货币幻觉 (Money Illusion):} 长期来看,菲利普斯曲线是垂直的;短期内,只有非预期的通胀才能降低失业率。
\end{itemize}

\subsubsection{2. 理性预期革命 (Rational Expectation Revolution)}
宏观经济学将“预期”置于舞台中心。

适应性预期 (Adaptive Expectation): 人们是“向后看”的 ($\pi_t^e = \pi_{t-1}$)。缺陷在于如果通胀持续上升,预期总是存在系统性偏差。

理性预期 (Rational Expectation): 人们是“向前看”的,利用所有可用信息形成预期。这导致了政策无效性 (Policy Irrelevance):除非政策是意料之外的,否则货币政策无效。

\begin{alertbox}{卢卡斯批判 (Lucas’ Critique)}
\textbf{观点:} 主流凯恩斯模型的政策分析是无效的,因为它们没有认真对待“预期”。
\textbf{逻辑:} 计量模型估算的参数(如 $c = \alpha + \beta Y$ 中的 $\beta$)并非一成不变。一旦政策改变,人们的预期和行为模式改变,参数 $\beta$ 也会失效。
\textbf{结论:} 我们需要对包含深层参数 (Deep parameters) 的结构进行建模,而不是依赖历史数据的简化形式系数。
\end{alertbox}

\subsubsection{3. 真实商业周期 (RBC)}
Kydland \& Prescott 提出:
\begin{itemize}
    \item \textbf{Money doesn't matter:} 价格瞬间调整,货币中性。
    \item \textbf{Technology shocks:} 技术冲击是短期波动的根源。
    \item \textbf{Optimal choices:} 经济波动是人们对技术冲击做出的最优选择反应,不需要政府干预。
\end{itemize}

\subsection{新凯恩斯主义 (New Keynesian Models)}
新古典-凯恩斯综合 (Neoclassical-Keynesian Synthesis): 结合了微观基础与市场不完美性。

\begin{itemize}
    \item \textbf{Market Imperfection:} 引入 菜单成本 (Menu costs) 和 价格粘性 (Price stickiness)。
    \item \textbf{解释:} 即使人们是理性的,由于调整价格有成本,价格和工资不会瞬间调整以出清市场。
    \item \textbf{结论:} 货币政策在提高经济效率方面可以发挥重要作用。
\end{itemize}

\end{document}